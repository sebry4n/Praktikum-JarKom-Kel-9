\section{Pendahuluan}
\subsection{Latar Belakang}
Bagian ini menjelaskan alasan dilaksanakannya praktikum, termasuk pentingnya topik yang dibahas. Latar belakang mencantumkan permasalahan yang ingin diselesaikan, urgensi pembelajaran topik, serta keterkaitannya dengan aplikasi dunia nyata atau teknologi saat ini.

\subsection{Dasar Teori}
Bagian ini memuat teori-teori dasar yang mendukung pelaksanaan praktikum. Penjelasan mencakup konsep teknis, nama istilah, serta prinsip ilmiah yang relevan. Tujuannya adalah untuk memberikan pemahaman mendalam sebelum praktikum dilakukan.

%===========================================================%
\section{Tugas Pendahuluan}
Bagian ini berisi jawaban dari tugas pendahuluan yang telah anda kerjakan, beserta penjelasan dari jawaban tersebut
\begin{enumerate}
	\item Jika kamu ingin mengakses web server lokal (IP: 192.168.1.10, port 80) dari jaringan luar, konfigurasi NAT apa yang perlu kamu buat?
	\item Menurutmu, mana yang lebih penting diterapkan terlebih dahulu di jaringan: NAT atau Firewall? Jelaskan alasanmu.
	\item Apa dampak negatif jika router tidak diberi filter firewall sama sekali?
\end{enumerate}