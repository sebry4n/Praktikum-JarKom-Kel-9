\section{Pendahuluan}
\subsection{Latar Belakang}
Pada awal perkembangan teknologi jaringan komputer, semua perangkat harus dihubungkan menggunakan kabel. Hal ini kemudian menjadi kendala, terutama ketika jarak antar perangkat berjauhan atau saat pengguna ingin berpindah tempat dengan leluasa. Instalasi kabel yang rumit, mahal, dan tidak fleksibel membuat penggunaan jaringan kabel menjadi kurang efisien, terutama di lingkungan yang dinamis seperti sekolah, kantor, atau rumah modern. Untuk mengatasi keterbatasan ini, mulailah dikembangkan jaringan tanpa kabel (nirkabel) atau disebut juga wireless network. Solusi awal jaringan nirkabel muncul dalam bentuk teknologi Wi-Fi, yang memungkinkan perangkat seperti laptop dan smartphone terhubung ke internet melalui sinyal radio. Meskipun pada awalnya kecepatan dan jangkauannya terbatas, teknologi ini menjadi sangat populer karena memberikan kebebasan dan kenyamanan bagi penggunanya. Seiring waktu, teknologi wireless terus berkembang. Kecepatan transfer data meningkat drastis, jangkauan sinyal menjadi lebih luas, dan keamanan jaringan semakin diperkuat. Selain Wi-Fi, muncul juga teknologi lain seperti Bluetooth, wireless bridge, serta jaringan mesh yang memungkinkan cakupan Wi-Fi meluas tanpa mengurangi stabilitas koneksi. Saat ini, jaringan nirkabel telah menjadi bagian penting dalam kehidupan sehari-hari. Hampir semua tempat umum menyediakan akses Wi-Fi, dan banyak perangkat modern yang dirancang hanya untuk bekerja secara nirkabel, misalnya tablet wifi only. Perkembangan terbaru bahkan mengarah pada penggunaan Wi-Fi 6 dan Wi-Fi 7, yang menawarkan kecepatan sangat tinggi dan efisiensi dalam lingkungan dengan banyak perangkat. Dengan teknologi ini, jaringan nirkabel bukan hanya alternatif, tetapi sudah menjadi standar utama dalam membangun jaringan modern yang cepat, fleksibel, dan mudah digunakan.

\subsection{Dasar Teori}
Jaringan wireless atau jaringan nirkabel adalah jenis jaringan komputer yang memungkinkan perangkat untuk saling terhubung dan bertukar data tanpa menggunakan kabel fisik, melainkan melalui gelombang radio atau sinyal inframerah. Dengan jaringan ini, perangkat seperti laptop, smartphone, dan tablet dapat terhubung ke internet atau jaringan lokal dengan lebih fleksibel walau pada mobilitas tinggi. Poin utama dari penggunaan jaringan wireless adalah penggunaan frekuensi radio sebagai media transmisi data, menggantikan kabel tembaga atau serat optik, yang mahal dan rumit. Komunikasi ini biasanya dilakukan melalui perangkat seperti access point, router wireless, atau wireless adapter. Jaringan wireless dapat dibedakan menjadi beberapa jenis berdasarkan cakupan dan fungsinya. Jenis-jenis tersebut antara lain Wireless Personal Area Network (WPAN) yang jangkauannya sangat dekat, biasanya digunakan untuk koneksi antar perangkat pribadi seperti Bluetooth dan NFC. Selanjutnya ada Wireless Local Area Network (WLAN), yang umumnya digunakan di rumah atau kantor menggunakan teknologi Wi-Fi. Untuk cakupan yang lebih luas seperti dalam satu kota, terdapat Wireless Metropolitan Area Network (WMAN), contohnya adalah WiMAX. Adapun Wireless Wide Area Network (WWAN) mencakup area sangat luas. Umumnya ia menggunakan jaringan seluler seperti 3G, 4G, atau 5G. Jika dibandingkan dengan jaringan berkabel (wired), jaringan wireless memiliki beberapa perbedaan. Dalam hal media transmisi, jaringan kabel menggunakan kabel fisik seperti Ethernet atau fiber optik, sedangkan jaringan wireless menggunakan gelombang radio. Dari sisi mobilitas, jaringan wireless jauh lebih fleksibel karena tidak membutuhkan kabel, sehingga memungkinkan pengguna berpindah tempat dengan mudah. Instalasi jaringan wireless juga relatif lebih sederhana dan cepat dibandingkan jaringan kabel. Namun, jaringan kabel umumnya memiliki kecepatan dan kestabilan yang lebih tinggi serta lebih aman dari segi fisik karena akses jaringan hanya dapat dilakukan melalui koneksi kabel. Oleh karena dalam jaringan wireless sinyal dapat diakses siapa saja dalam jangkauan, keamanan dalam jaringan wireless menjadi hal yang sangat penting. Beberapa ancaman yang sering terjadi adalah penyadapan (eavesdropping), penyusupan (spoofing), dan akses unaouthorized.Karena itu, berbagai langkah keamanan diterapkan seperti penggunaan enkripsi WPA2 atau WPA3, pengaturan sandi Wi-Fi yang kuat, pembatasan akses melalui MAC address, serta menyembunyikan SSID. Jaringan wireless menggunakan berbagai protokol penting. Salah satunya adalah standar IEEE 802.11, yang mengatur komunikasi pelbagai varian Wi-Fi, seperti 802.11a/b/g/n/ac hingga Wi-Fi 6 (802.11ax). Selain itu untuk mengenkripsi data, digunakan protokol keamanan WPA (Wi-Fi Protected Access). Ada juga protokol TCP/IP yang berperan dalam pengiriman data antar perangkat, dan DHCP digunakan untuk memberikan alamat IP secara otomatis. Untuk keamanan data saat berselancar di internet, protokol seperti HTTPS dan SSL/TLS juga digunakan.

%===========================================================%
\section{Tugas Pendahuluan}
\begin{enumerate}
	\item Jelasin apa yang lebih baik, jaringan wired atau jaringan wireless? \\
	Baik jaringan wired atau wireless memiliki kekurangan dan eklebihannya masing masing. Jadi, mana yang lebih baik tergantung kasusu penggunannya. Misal, untuk data center atau server yang memerlukan keamanan tinggi dan latensi serendah mungkin, jaringan wired ebih baik untuk digunakan. Sedangkan untuk penggunaan rumahan dengan perangkat mobile (smartphone, laptop, dsb), jaringan wireless lebih efisien dan mudah untuk digunakan.
	\item Apa perbedaan antara router, access point, dan modem?\\
	Modem (modular-demodulator) merupakan perangkat yang mengubah sinyal ISP menjadi sinyal digital. Sederhananya, ia berfungsi agar perangkat bisa terhubung ke internet dari ISP. Sedangakn router adalah perangkat yang digunakan untuk mendistribusikan network ke dalam jaringan lokal (LAN). Router tidak langsung terhubung ke internet, ia harus dihubungkan dengan modem atau kabel yang memiliki jaringan internet lain. Terakhir access point adalah perangkat yang digunakan untuk menyebarkan internet secara wireless. Ia juga tidak langsung terhubung ke internet dan membutuhkan router atau switch. Jadi, perbedaan ketiganya adalah fungsinya dalam menyebarkan interent. Modem untuk menghubungkan internet dari ISP ke jaringan lokal, router untuk menyebarkan jaringan lokal melalui kabel, dan access point untuk menyebarkan jaringan secara wireless.
	\item Jika kamu diminta menghubungkan dua ruangan di gedung berbeda tanpa menggunakan kabel, perangkat apa yang kamu pilih? Jelaskan alasannya.\\
	Wireless Bridge, karena perangkat ini dapat mengextend jaringan untuk jarak yang berjauhan dengan lebih reliable berkat mekanisme point to point nya.
\end{enumerate}